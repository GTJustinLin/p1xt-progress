\chapter{Course: Coursera Calculus 1}
\section{Week 1}
No content, just a quick "this is how this course is going to work."

\section{Week 2}
\subsection{Functions}

A function assigns to each number in it's domain another number. A function must produce a single output value for each input value. Every time a particular input value is given, the function must produce the exact same output.

A domain is the set of numbers to which a function applies.

Two functions are the same if they apply to the same domain and produce the same value for each input in the domain.

When you compose two functions, the output from the first becomes the input to the second.

The limit of a sum is the sum of the limits (provided the limits exist).

The limit of a product is the product of the limits (provided the limits exist).

Squeeze Theorem 

\subsection{Limits}

\section{Week 3}
\section{Week 4}
\section{Week 5}
\section{Week 6}
\section{Week 7}
\section{Week 8}
\section{Week 9}
\section{Week 10}
\section{Week 11}
\section{Week 12}
\section{Week 13}
\section{Week 14}
\section{Week 15}
\section{Week 16}

\section{Problem Set 1}

\begin{problem}{1A - 1b}
\end{problem}
\begin{solution}
\end{solution}
